%Copyright 2016 R.D. Martin
%This book is free software: you can redistribute it and/or modify it under the terms of the GNU General Public License as published by the Free Software Foundation, either version 3 of the License, or (at your option) any later version.
%
%This book is distributed in the hope that it will be useful, but WITHOUT ANY WARRANTY; without even the implied warranty of MERCHANTABILITY or FITNESS FOR A PARTICULAR PURPOSE.  See the GNU General Public License for more details, http://www.gnu.org/licenses/.
\chapter{Statistics - Presenting Data}
\label{Chap:statData}
\section{Data and Statistics}

Statistical analysis is useful in many field, including physics, because virtually any measurement that is performed multiple times will show a variation, either due to the imperfection of the measuring apparatus, because of variations in the samples, or because the quantity truly has a variance (e.g. the speed of gas molecules at a specific temperature and pressure).

For example, if you wanted to model the length of 8 month old cats and set out to measure cats, you would get a variation in your measurements: you would get variations between different cat breeds, you would get variations because the cats will have different lengths of hair, you would get variations because some of the cats would move more than others while you try to measure them, and most likely, you will get variations because not all 8 month old cats are identical. Even if you tried to minimize systematic effects (e.g. by shaving and sedating all the cats and working with a single breed), you will still get variations. 

The goal of statistics would be to understand if your measurements allow you to infer something about a model for cat length, perhaps that 8 month old cats are on average L +/- dL long. Maybe you would find that the average cat length is different between countries, or different types of cat owners. As a scientist, you want to be assured that you can make rigorous statements about your findings and rigorously test different cat length models even when the data are expected to show a spread.

For the specific case of many measurements of a single value (e.g cat length), there are several common quantities that can be calculated to provide some description of the set (i.e. the distribution) of measurements that were obtained. We start by introducing those quantities (such as the mean and standard deviation), and then describe the use of the histogram to visualize data of a statistical nature.

\section{Basic statistical parameters}

\subsection{Mean and variance}

The mean and variance are the most common quantities to characterize a distribution of measurements. If a set of $N$ measurements, \{$x_1, x_2, \dots,x_N$\}, of a single quantity were performed, the ``sample mean'' is defined as:
\begin{align}
 \bar{x} \equiv \frac{1}{N} \sum_{i=1}^{i=N} x_i  
\end{align}
which is simply the algebraic average. The ``sample variance'', $\sigma_x^2$, is defined as:
\begin{align}
 \text{var}\equiv \sigma_x^2 = \frac{1}{N-1} \sum_{i=1}^{i=N} (x_i-\bar{x})^2 
\end{align}
and the ``sample standard deviation'', $\sigma_x$, is defined as the square root of the sample variance:
\begin{align}
 \sigma_x\equiv\sqrt{\text{var}}
\end{align}
While the sample mean gives the average value of the measurement, the sample variance and standard deviation indicate the average distance between the measurements and the sample mean. That is, whether the measured values are all really close to the sample mean, or whether they are spread out. The sample standard deviation is also a representative measure of the uncertainty in a single measurement. 

Note the $(N-1)$ in the definition of the sample variance, instead of $N$. This is a subtlety that arises from the fact that the formula for the sample variance uses the sample mean as determined from the data. If $N$ is large, there will not be a big difference in the result from including the correct $N-1$.

This subtlety arises due to the fact that the ``true'' mean is not known, rather it has been estimated from the data. One way to think of this is that there is a large ``population'' of values, and we have only measured a small sample of them (for example, you would measure a small subset of cats and try to infer something about all cats). It is likely that, by chance, the sample mean that you measured is not exactly the population mean. We think of the sample mean as an ``estimator'', also referred to as a ``statistic'', whereas the true population mean can be referred to as a ``parameter'' of the population (which you are trying to estimate from data). The ``population'' may represent an actual group of which you have measured a subset, or it could be a theoretical model. That is, your sample mean could be an estimator for the mean of a model that you are using to describe the data.

Similarly, the sample variance is only an estimator for the true variance in the population. It has been shown (Bernoulli's correction), that dividing by N instead of $(N-1)$ would yield a sample variance that is systematically smaller than the true population variance (we would call this a ``biased estimator''). Dividing by $(N-1)$ removes the bias. 

Note that if the true mean of the population, $\mu$, is known, then the sample variance as an estimator of the population variance is given by:
\begin{align}
 \text{var}\equiv \sigma_x^2 = \frac{1}{N} \sum_{i=1}^{i=N} (x_i-\mu)^2 
\end{align}
where we now divide by $N$ since we did not use an estimate for the mean of the population.
 
\subsection{Error on the mean} 
The variance and standard deviation are measurements of how spread out the measurements are relative to the mean. We can also determine how well the mean is determined from our sample by defining the ``error on the mean'', $\sigma_{\bar x}$, which is given by the standard deviation divided by the square root of the number of measurements:
\begin{align}
 \sigma_{\bar x}= \frac{\sigma_x}{\sqrt N}
\end{align}
We will prove this result in Chapter \ref{chap:StatsNormal}, but we can observe here that it makes intuitive sense. As we gather more samples, then the error on the mean decreases (as the square root of the number of samples), which makes sense. 
 
\subsection{Mode and Median}

The mode and median are also of use to describe data that have a distribution of values. The mode is the value that occurs the most often (sometimes also called the "most probable value"). The median of the data set is the value for which there are an equal number of values that are bigger and smaller. For a set of measurements that are normally (gaussian) distributed (as described in Chapter \ref{chap:StatsNormal}), the mean, mode and median overlap. In general, all three parameters can be different. 

\begin{example}{}{Calculate the mean, variance, error on the mean, mode and median for the following set of measurements: \{1.1, 1.2, 1.0, 1.3, 1.4, 1.5, 1.2, 1.3, 1.3, 1.1, 1.0, 1.6 \}}{}
\label{ex:meanvardata}
We can compute these quantities using a simple python program:
\begin{lstlisting}[frame=single] 
import numpy as np
import scipy.stats as stats
from math import *
#Copy the data from the book into a numpy array
xi=np.array([1.1, 1.2, 1.0, 1.3, 1.4, 1.5, 1.2, 1.3, 1.3, 1.1, 1.0, 1.6])
N=xi.size
#Use the numpy mean() function to get the mean:
xmean=xi.mean()
print("mean: {:.2f}".format(xmean))
#The variance (note that we have to specify ddof=1, which results in using the 
# N-1 in the denominator instead of N)
xvar=xi.var(ddof=1)
print("variance: {:.3f}".format(xvar))
#The standard deviation is either the square root of the variance, or we can
#get it from numpy
xstd=xi.std(ddof=1)
print("std. dev.: {:.3f}".format(xstd))
#The error on the mean
xmeanerr=xstd/sqrt(N)
print("error on the mean: {:.3f}".format(xmeanerr))

#Sort the array to make it easier to see the mode and median:
xi_sorted = np.sort(xi)
print("Sorted:",xi_sorted)
#The median can be found by:
xmedian=np.median(xi)
print("The median is:",xmedian)
#The mode must be found using the scipy.stats module
xmode,modecount=stats.mode(xi)
print("The mode is:",xmode[0])
\end{lstlisting}
The output is:
\begin{verbatim}
mean: 1.25
variance: 0.035
std. dev.: 0.188
error on the mean: 0.054
Sorted: [ 1.   1.   1.1  1.1  1.2  1.2  1.3  1.3  1.3  1.4  1.5  1.6]
The median is: 1.25
The mode is: 1.3
\end{verbatim}

Note that by default, the \code{numpy} module uses $N$ in the denominator instead of $N-1$ when calculating the sample variance and standard deviation. One needs to specify the ``delta degrees of freedom'' parameter, \code{ddof}, which is the number to subtract from $N$ (in our case it's 1).

\end{example}

\subsection{Moments, Skewness and Kurtosis}

The skewness and kurtosis are higher order "moments" of the data that describe how symmetric the data are with respect to the mode (skewness) and whether the data are grouped tightly at the mode (kurtosis).

The $\alpha$ ``sample moment'' of the data is defined as:
\begin{align}
 m_\alpha=\frac{1}{N-1} \sum_{i=1}^{i=N} (x_i-\bar{x})^\alpha 
\end{align}
The sample variance is thus the ``second'' sample moment. The skewness is defined as:
\begin{align}
 \text{skew} = \frac{m_3}{m_2^{3/2}} 
\end{align}
The skewness will be positive if a larger fraction of the measurements are bigger than the mode. It will be negative if most of the measurements are smaller that the mode. Effectively it tells you whether the mean is to the right or the left of the mode. If the skewness is 0, then mean and mode coincide.

The kurtosis is defined as:
\begin{align}
  \text{kurt}  =\frac{m_4}{m_2^{2}} 
\end{align}
If the data are normally distributed about the mean, then the kurtosis is equal to 3. If the data have a kurtosis higher than 3, then the distribution has a sharper peak than a normal distribution. The smallest kurtosis is 1, corresponding to data that are not concentrated about any specific value.

\subsection{Covariance and correlation}
Often, we may measure two different quantities simultaneously multiple times. For example, we may measure both the length and the weight of 8 month old cats, and we probably would expect that the two are ``correlated''; that is, that if one quantity is bigger, then the other one tends to be bigger as well. We can quantify how correlated two quantities are by measuring the covariance and the correlation factors. 

Let us suppose that we have two quantities, $x$ and $y$, that we measure simultaneously $N$ times, giving us two sets of measurements  $x_i=\{x_1, x_2,\dots, x_N\}$ and $y_i=\{y_1, y_2,\dots, y_N\}$. The covariance, $\sigma_{xy}$, is defined to be:
\begin{align}
\sigma_{xy}\equiv\frac{1}{N-1}\sum_{i=1}^{i=N}(x_i-\bar x)(y_i-\bar y)
\end{align}
where $\bar x$ and $\bar y$ are the means of the measurements of $x$ and $y$, respectively. The $N-1$ in the denominator has the same origin as when calculating the variance, and is required to recover the correct formula for the variance if $y=x$. The covariance is thus positive when $x$ and $y$ tend to both be larger or smaller than their mean at the same time (for example $x_i$ bigger than $\bar x$ means that $y_i$ is likely bigger than $\bar y$). The covariance is negative when $x$ and $y$ tend to be in opposite directions from their means at the same time (e.g. $x_i$ bigger than $\bar x$ leads to $y_i$ being smaller than $\bar y$). The covariance is near 0 when a value of $x$ bigger or smaller than $\bar x$ does not lead to a value of $y$ that is predictably bigger or smaller than $\bar y$. When the covariance is positive we say that $x$ and $y$ are correlated, whereas when it is negative, we say that they are ``anti-correlated''.

The correlation factor, $\rho_{xy}$, sometimes also called $r$ or $R$ or ``the r-factor'', is simply the covariance divided by the standard deviations of $x$ and $y$, $\sigma_x$ and $\sigma_y$, respectively. This leads to $\rho_{xy}$ being bound between -1 and 1. A correlation factor of 1 means that $x$ and $y$ are perfectly correlated, a value of 0 indicates that they are un-correlated, and a value of -1 that they are perfectly anti-correlated. The correlation factor, $\rho_{xy}$ is given by:
\begin{align}
\rho_{xy}\equiv\frac{\sigma_{xy}}{\sigma_x\sigma_y}
\end{align}

\begin{example}{}{Calculate the covariance and correlation factors for the following 12 measurements of the quantities $x$ and $y$: $x_i=\{1.1, 1.2, 1.0, 1.3, 1.4,1.5, 1.2, 1.3, 1.3, 1.1, 1.0, 1.6\}$ and $y_i=\{1.3, 0.9, 0.9, 1.0, 1.6, 1.1, 1.2, 1.2, 1.2, 0.9, 0.9, 1.1\}$}{}
\label{ex:covariance}
The quantities can easily be evaluated in python using the following code:
\begin{lstlisting}[frame=single] 
import numpy as np
import pylab as pl

#copy the values into numpy arrays:
xi=np.array([1.1, 1.2, 1.0, 1.3, 1.4,1.5, 1.2, 1.3, 1.3, 1.1, 1.0, 1.6])
yi=np.array([1.3, 0.9, 0.9, 1.0, 1.6, 1.1, 1.2, 1.2, 1.2, 0.9, 0.9, 1.1])
#get the mean and standard deviations:
xmean=xi.mean()
ymean=yi.mean()
xstd=xi.std(ddof=1)
ystd=yi.std(ddof=1)
#calculate the covariance and correlation:
covxy=((xi-xmean)*(yi-ymean)).sum()/(xi.size-1)
corxy=covxy/xstd/ystd
#make some text with the results
text='''
Covariance: {:.2f}
Correlation: {:.1f}'''.format(covxy,corxy)
#plot these as a scatter plot, including the text
pl.scatter(xi,yi)
pl.xlabel('x')
pl.ylabel('y')
pl.title("scatter plot of x and y")
pl.text(0.95,1.5,text,fontsize=14)
pl.show()
\end{lstlisting}
The above code determined the covariance to be 0.02 and the correlation to be 0.42. The values of $x$ and $y$ are thus correlated, and on average, a bigger value of $x$ will result in a larger value of $y$. A scatter plot of $x$ and $y$ is show in Figure \ref{fig:scatter_cov}, highlighting the trend in the two correlated variables.

\capfig{0.5\textwidth}{figures/scatter_cov.png}{\label{fig:scatter_cov} Scatter plot of $x$ and $y$ values for different measurements (example \ref{ex:covariance}), showing the correlation between the two quantities.} 

\end{example}

The correlation factor is a measure of whether two quantities appear to be related in some manner. For example, a positive correlation between $x$ and $y$ indicates that if one measures a high value of $x$, then one is also likely to measure a high value of $y$. This does not mean that a high value of $x$ \textit{causes} a high value of $y$. ``Correlation does not imply causation'' is a popular slogan in statistics (strongly emphasized in social studies). Measuring the correlation should never be confused with trying to model one variable as a function of the other. For example, it is likely that the length of cats is correlated with their weights, but a simple confirmation of a strong correlation factor in no way implies that the length of cats causes them to be heavier. For example, it could be measured that going to sleep with one shoe on is correlated with having a headache in the morning. It would be wrong to conclude that sleeping while wearing a shoe causes headaches. Perhaps, there is a third variable that leads to both observations, for example, going to bed intoxicated. 

We can ask ourselves whether a particular correlation is ``significant'', that is, whether there is strong evidence to support a claim that two variables are correlated. If we take purely random measurements of two quantities, $x$ and $y$, it is unlikely that their correlation factor would come out to be exactly 0. Rather, we would expect that as the number of measurements increases, their correlation would approach 0 as it becomes more precisely defined.  The question is thus to quantify the significance of a measured correlation. This can be done by considering the value of the correlation factor that is unlikely to be possible if the measurements are uncorrelated. Specifically, for $N$ measurements, we can determine a threshold in correlation for which it is unlikely for $N$ uncorrelated measurements to give a correlation that is higher than the threshold. A sample of these ``Pearson correlation'' threshold values are shown in Table \ref{tab:PearsonCorr}. For example, if 8 measurements are used to determine the correlation factor, if it is less than 0.621 it is not significant at the 95\% level. That is, there is a 5\% chance that 8 random uncorrelated measurements would have a correlation factor of 0.621 or bigger.

\begin{table}[h!]
\center
\begin{tabular}{|c|c|c|}
\hline
\textbf{N} & \textbf{5\%} &\textbf{2.5\%}\\
\hline
4 & 0.900 &0.95 \\
5 & 0.805 &0.878 \\
6 & 0.729 &0.811 \\
7 & 0.669 &0.754 \\
8 & 0.621 &0.707 \\
9 & 0.582 &0.666 \\
10 & 0.549 &0.632 \\
\hline
\end{tabular}
\caption{\label{tab:PearsonCorr} Threshold values for a correlation determined from $N$ measurements to be significant at the 95\% and 97.5\% levels. The values corresponds to either a positive or negative correlation. If the sign of the correlation is not important, these correspond to 90\% and 95\% significance.}
\end{table}


\section{The histogram}
The histogram is a way to visualize a set of measurements of the same quantity. With repeated measurements, we generally obtain a ``distribution'' of values and the histogram is a way to visualize that distribution. In example \ref{ex:meanvardata}, we had the following measurements: $x_i=$\{ 1., 1., 1.1, 1.1, 1.2, 1.2, 1.3, 1.3, 1.3, 1.4, 1.5, 1.6 \} for the value of some parameter, $x$. One way to visualize these data is to count how many times each value occurs and to make a table, as in Table \ref{tab:measfreq}
\begin{table}[!ht]
\center
\begin{tabular}{|c|c|}
\hline
value & frequency\\
\hline
1.0 &2\\
\hline
1.1 &2\\
\hline
1.2 &2\\
\hline
1.3 &3\\
\hline
1.4 &1\\
\hline
1.5 &1\\
\hline
1.6 &1\\
\hline
\end{tabular}
\caption{\label{tab:measfreq}Frequency of occurrence of each measurement.}
\end{table}

We can visualize this table easily using a ``histogram'', as in Figure \ref{fig:simplehist}.
\capfig{0.5\textwidth}{figures/simplehist.png}{\label{fig:simplehist}A simple histogram of the measurements from Example \ref{ex:meanvardata} and Table \ref{tab:measfreq}.}

Histograms are drawn as ``bar charts'', since measurements at different values on the x-axis are unrelated (it would not be meaningful to treat the columns in Table \ref{tab:measfreq} as x and y values for data points and connect them with a line). We call each bar on the chart a ``bin'', since it ``holds'' the number of times each measurement occurred. The histogram is a good way to visualize a distribution of values; in our case, we can see that the value 1.3 comes up the most, and that the values below 1.3 come up more often than the ones above (the distribution is asymmetric about the mode). With only 12 measurements, it is not too difficult to sort the values and come to these conclusions without drawing a histogram; in the case where we have many measurements, histograms become an extremely useful visualization (and analysis) tool.

In the case that we just illustrated, we counted the occurrences of specific values, such as 1.3 or 1.2. If we had measured these numbers with additional precision (more decimals), it is likely that they each would be different. In this case, it would not make sense to count how many times each specific value occurred, since each value would likely occur only once. What we really want to do, is count how many times we get a value that is within a certain range. That is, we define the left and right edge for each bin, and then count how many times we get a value within that range. The histogram in Figure \ref{fig:simplehist} was in fact defined to have 7 bins: the first bin from 0.95 to 1.05, the second bin from 1.05 to 1.15, etc, in such a way that the numbers 1.0, 1.1, etc ended up at the center of each bin. In Figure \ref{fig:simplehist_rebin}, we have made a histogram of the same data, but using only three, unequally-sized, bins: from 0.9 to 1.1, from 1.1 to 1.4, and from 1.4 to 1.7, which results in an equally valid histogram of the data. 
 
\capfig{0.5\textwidth}{figures/simplehist_rebin.png}{\label{fig:simplehist_rebin}The same histogram as in Figure \ref{fig:simplehist} but with different size bins.}

As you can see, we have some liberty in how we define the ``binning'' of our histogram. If we make all of the bins really narrow, then they would typically have counts of either 0 or 1 and we would not see much structure in the data. Conversely, if we make the bins very wide, then all the measurements will end up into one bin, and the histogram will not help us to  visualize anything either.

In general, we need to play with the binning for plotting data so that the histogram looks reasonable. A good rule of thumb is that the bins should have at least 5 counts in them. Ideally, we should also try to have equally sized bins (rather that varying bin widths as in Figure \ref{fig:simplehist_rebin}), as it makes it easier to analyse the histogram. Refer to section \ref{subsub:pythonhist} for all of the commands to make histograms in python, as well as the code at the end of this section.

It is sometimes useful to ``normalize'' a histogram such that the area of the bins of the histogram sum to 1. If each bin, $i$, contains $n_i$ counts (the bin height) and has a bin width, $dx_i$, then the area under a histogram, $A$, is simply given by summing the area of each rectangular bin:
\begin{align}
A = \sum_{i=1}^{i=N}n_idx_i
\end{align}
In order to normalize a histogram, we must thus divide each bin content by $A$. If all of the bins in the histogram have equal width, $dx$, (say all $dx_i=dx$), then the area of the histogram is given by:
\begin{align}
A = dx\sum_{i=1}^{i=N}n_i
\end{align}
where $\sum_{i=1}^{i=N}n_i$ is equal to the number of measurements (the sum of the contents of each bin). In Figure \ref{fig:simplehist} from above, we had a fixed bin width of 0.1 and a total of 12 measurements; the area is thus given by $A=1.2$. We can thus ``normalize'' that histogram by dividing the content of each bin by 1.2, as in Figure \ref{fig:simplehist_normed}. In python, histograms can easily be normalized by passing the argument \code{normed=True} to the \code{hist()} function. 

\capfig{0.5\textwidth}{figures/simplehist_normed.png}{\label{fig:simplehist_normed}The same histogram as in Figure \ref{fig:simplehist} but normalized to have an area of 1.}

As we will see in the next chapters, a normalized histogram can be thought of as a ``probability density function''. If we assume that the data in the histogram are representative of the distribution of measurements that we would get if we repeated the measurements, then the normalized bin contents are representative of the probability of obtaining a certain measurement that would fall in that bin. For example, using Figure \ref{fig:simplehist_normed}, we could quote that the probability of obtaining a measurement between 1.25 and 1.35 is given by (bin content)$\times$ (bin width) = 2.5 $\times$ 0.1 = 25\%. In our original 12 measurements, we had obtained the number 1.3 three times, or 25\% of the time.

The example python code below shows how to obtain the histograms shown in this section:
\begin{lstlisting}[frame=single] 
import numpy as np
import pylab as pl
#Make the histograms in Figures 5-2, 5-3, 5-4
#The data:
xi=np.array([1.1, 1.2, 1.0, 1.3, 1.4,1.5, 1.2, 1.3, 1.3, 1.1, 1.0, 1.6])

#Make a histogram with equal bin width and plot it:
#Here we specify the edges of the bins:
pl.hist(xi, bins=[0.95,1.05,1.15,1.25,1.35,1.45,1.55,1.65],color='gray')
pl.axis([0.9,1.7,0,4])
pl.grid()
pl.xlabel("Measurement")
pl.ylabel("Number of occurrences")
pl.show()

#Use 3 unequal bins which we specify
#Also get the number of counts in each bin:
n,bins,patches=pl.hist(xi, bins=[0.9,1.1,1.4,1.7],color='gray',normed=False)
pl.axis([0.9,1.7,0,8])
pl.grid()
pl.xlabel("Measurement")
pl.ylabel("Number of occurrences")
pl.show()
print("Bins used:",bins)
print("Counts in each bin",n)

#Use original binning, but normalize the histogram:
pl.hist(xi, bins=[0.95,1.05,1.15,1.25,1.35,1.45,1.55,1.65],color='gray', normed=True)
pl.axis([0.9,1.7,0,4])
pl.grid()
pl.xlabel("Measurement")
pl.ylabel("Normalized number of occurrences")
pl.title("Normalized histogram")
pl.show()
\end{lstlisting}
The output is:
\begin{verbatim}
Bins used: [ 0.9  1.1  1.4  1.7]
Counts in each bin [ 2.  7.  3.]
\end{verbatim}

\section{Comparing statistical data to a model}
We now have a few tools that can allow us to compare statistical data to a model (or between different data sets). For example, if one researcher has measured the lengths of 8 month old Canadian sphynx cats and another researcher has measured the lengths of 8 month old American sphynx cats, we can think about whether the Canadian and American cats are statistically similar. We could compare the mean lengths of the two samples using the corresponding errors on the mean to see if they agree. If the means agree within their respective errors on the mean, we could still conclude that Canadian and American cats are different if the shapes of the normalized histograms of the measurements were very different (e.g. you could get the same mean, but vastly different distributions).

\section{Summary}
In this chapter, we covered a few useful metrics to characterize a set of $N$ measurements, \{$x_1, x_2, \dots,x_N$\}, of a single quantity. In particular, we defined the mean:
\begin{align}
 \bar{x} \equiv \frac{1}{N} \sum_{i=1}^{i=N} x_i 
\end{align}
The ``sample variance'':
\begin{align}
 \text{var}\equiv \sigma^2 = \frac{1}{N-1} \sum_{i=1}^{i=N} (x_i-\bar{x})^2 
\end{align}
and the ``sample standard deviation'': 
\begin{align}
 \sigma\equiv\sqrt{\text{var}}
\end{align}
We also introduced the error on the mean:
\begin{align}
 \sigma_{\bar x}= \frac{\sigma}{\sqrt N}
\end{align}

If we have a set of simultaneous measurement of two quantities, $x_i=\{x_1, x_2,\dots, x_N\}$ and $y_i=\{y_1, y_2,\dots, y_N\}$, the covariance, $\sigma_{xy}$, is defined to be:
\begin{align}
\sigma_{xy}\equiv\frac{1}{N-1}\sum_{i=1}^{i=N}(x_i-\bar x)(y_i-\bar y)
\end{align}
where $\bar x$ and $\bar y$ are the sample means of the measurements of $x$ and $y$, respectively. The correlation factor, $\rho_{xy}$ is defined to be:
\begin{align}
\rho_{xy}\equiv\frac{\sigma_{xy}}{\sigma_x\sigma_y}
\end{align}
where $\sigma_x$ and $\sigma_y$ are the sample standard deviations of the measurements of $x$ and $y$, respectively. The correlation is a measure of whether the quantities $x$ and $y$ tend to be simultaneously big or small when measured.

We finished by introducing the histogram as a way to visualize the distribution of measurements of a single quantity. We saw that we have liberty in how we define the location and sizes of the bins. We saw that by normalizing the area of the histogram, we could approximate a probability density function for the data.