%Copyright 2016 R.D. Martin
%This book is free software: you can redistribute it and/or modify it under the terms of the GNU General Public License as published by the Free Software Foundation, either version 3 of the License, or (at your option) any later version.
%
%This book is distributed in the hope that it will be useful, but WITHOUT ANY WARRANTY; without even the implied warranty of MERCHANTABILITY or FITNESS FOR A PARTICULAR PURPOSE.  See the GNU General Public License for more details, http://www.gnu.org/licenses/.
\chapter{Statistics - The Monte Carlo Method}
\label{Chap:statMonteCarlo}
The Monte Carlo (MC) method was developed to approximate difficult calculations by using random numbers to try and sample the solution. Originally, it was developed at the Los Alamos National Laboratory to simulate the propagation of neutrons in the design of nuclear weapons without having to solve the complex transport equations for neutrons. It has now gained widespread use in a variety of fields beyond physics, including economics, biology, and others. The Monte Carlo method is named after the Monte Carlo casino in Monaco (where James Bond is a regular), because the use of random numbers can remind one of using dice (or other instruments of chance) to perform calculations.

The basic idea behind the Monte Carlo method is to randomly sample the solution space in a clever way that does not require us to calculate the actual distribution of the solutions. For example, you may be interested in the distribution of profits that your poutine restaurant will make based on the number of customers that you get. Since every day you will get a random number of customers, ordering a random number of items, it is very difficult to calculate the distribution of the expected returns per day. With a Monte Carlo simulation, you would generate a random number of customers each day, generate random items to buy, calculate the profit, and then repeat many times. If you know the average distribution of the number of customers and their purchases, your simulation will yield the correct distribution for your expected returns.

\section{Random numbers from a computer}
Monte Carlo simulations require computers to generate random numbers. However, it is not possible to write a computer program that will generate truly random numbers, since computers will always execute a program in a predictable way. For that reason, the random numbers generated by a computer are called ``pseudo random numbers''. Generally, the peudo random number generating (PRNG) algorithms are based on taking a ``seed'' number and then generating a sequence of numbers that is seemingly random (although completely predictable if the seed number and the algorithm are known). 

The sequence will ultimately repeat itself and has a ``period'' that depends both on the seed and the algorithm. A good PRNG will have a large period and will generate random numbers that are ``uniform''. By convention, most random number generators return a floating point value between 0 and 1, that is uniformly distributed between 0 and 1 (that is, no number is more likely than any other). Finally, a good PRNG should produce numbers that are ``uncorrelated'', so that without knowing the algorithm, one would not be able to predict the next number in the sequence. For example, a PRNG could have a large period, produce numbers that are unformly distributed, but alternately produce large and small numbers, so that the sequence is correlated and predictable. If one is relying on a PRNG to produce numbers for a Monte Carlo simulation, and the PRNG is not reliable, the results of the calculation will be incorrect.

Apart from their use in Monte Carlo simulations, PRNGs are used widely in encryption, where very large random numbers are needed to produce encryption keys that are difficult to break. They are also used in a variety of other applications, including electronic lottery machines.

A very simple class of random number generators are the "Linear Congruential Generators" (LCG). They are given by the simple recursive relation:
\begin{align*}
 N_i = (aN_{i-1} + b) \; mod \; M 
\end{align*}

where $N_0$ is the seed of the sequence, and ``mod'' is the modulo operation \footnote{The modulo operation gives the remainder of a division. For example, 11 mod 3 is 2; the remainder of dividing 11 by 3 is 2.}. The integer constants $a$, $b$, $M$, determine the properties of the random numbers. $a$ is called the ``multiplier'', $b$ the ``increment'', and $M$ the ``modulus''. In order to obtain numbers, $u_i$, between 0 and 1, one divides $N_i$, by $M$:
\begin{align*}
 u_i = \frac{N_i}{M}
\end{align*}

The period of this class of PRNGs depends on the choice of constants and is at most $M$ ($N_i$ will be between 0 and $M-1$, so there are only $M$ possible values that $N_i$ can take).

A reasonable choice is Park and Miller's ``Minimal Standard'', with $a=16807$, $b=0$, and $M=2^{31}-1=2147483647$, which passes most stringent tests for a PRNG. A particularly terrible choice is the so-called RANDU generator, $a=65539$, $b=0$, $M=2^{31}$. This last choice was implemented on IBM machines for many years and has led to a plethora of inaccurate results (and probably still undiscovered ones).  

The following python code is a simple implementation of the LCG, which used the Park Miller minimal standard by default. Note the use of a ``function attribute'', \code{LCG.seed}, to store a number that the \code{LCG()} function updates (it starts with the seed of the sequence, and then gets updated with the latest value of $N_i$ to generate a new number the next time that it is called). A histogram of the values generated is shown in Figure \ref{fig:LCG}.
\begin{lstlisting}[frame=single] 
import numpy as np
import pylab as pl

#Define a function that returns LCG (pseudo) random numbers
def LCG (a=16807,b=0,M=2147483647):
    N1 = (a*LCG.seed+b) % M
    #print(LCG.seed)
    LCG.seed = N1
    return N1/M

#Initialize the seed:
LCG.seed=10135

#Generate an array of random numbers from the the LCG
nr=10000
r=np.zeros(nr)
for i in range(nr):
    r[i] = LCG(seed)

#Plot a histogram of the values:
pl.hist(r,bins=100,color='gray')
pl.xlabel('random number')
pl.ylabel('frequency')
pl.title('10,000 numbers for the Park-Miller LCG')
pl.show()
\end{lstlisting}

\capfig{0.5\textwidth}{figures/LCG.png}{\label{fig:LCG} Distribution of 10,000 numbers generated with the Park Miller Minimal Standard LCG.}

\subsection{Random numbers that follow a given distribution}
Although a good PRNG should be uniform, it is often desirable to generate random numbers according to a specific distribution. For example, one may want to generate numbers that are normally distributed rather than uniformly distributed. This can be done straightforwardly for any distribution using uniformly distributed random numbers.

If we have uniform random numbers, $x$, we wish to find a function, $y(x)$, such that the numbers, $y$, are distributed according to some chosen distribution, $g(y)$. Consider the cumulative distribution function (cdf), G(y), given by:
\begin{align*}
G(y) = \int_{-\infty}^y g(y)dy
\end{align*}

If we now choose the values of $y(x)$ such that $G(y)$ are uniform, then the values of $y$ will be distributed according to $g(y)$. If $x$ is a uniform random number, we can choose the $y$ such that:
\begin{align*}
x(y) = G(y)
\end{align*}
and invert the functions, such that:
\begin{align*}
y(x) = G^{-1}(x)
\end{align*}

The function inversion cannot usually be done analytically, although one can easily program a function inversion using interpolation from tabulated values. Also note that for distributions, $g(y)$ that are not monotonous, it may not be possible to use this method if the function inverse is not one-to-one. For example, if this method were used for the normal distribution, then it would only generate values on one side of the mean (but the algorithm could easily be modified to generate numbers that are symmetric about the mean).

As an example that can be done analytically, let us suppose that we want to generate random numbers that are distributed according to an exponential, $g(y) = e^{-\frac{y}{\tau}}$, we can analytically find the cdf, $G(y)$:
\begin{align*}
g(y) &= e^{-\frac{y}{\tau}}\nonumber\\
G(y) &= \int_{-\infty}^y g(y) = -\frac{1}{\tau} e^{-\frac{y}{\tau}}
\end{align*}
Setting $G(y)$ equal to $x$ and inverting:
\begin{align*}
x(y) &= G(y)
y(x) &= G^{-1}(x) = -\tau \ln{x} 
\end{align*}
That is, if we generate uniform random numbers, $x$, take their natural logarithm and multiply by $-\tau$, the resulting numbers will have a negative exponential distribution with decay constant $\tau$. This is illustrated in the following python code, which uses the uniform random numbers, \code{r}, generated in the code above with the LCG and converts them to numbers, \code{y}, that are exponentially distributed. The resulting histogram is shown in Figure \ref{fig:LCGexpo}.

\begin{lstlisting}[frame=single] 
#using the array of values, r, we transform these into values y:
tau = 10
y = -tau * np.log(r) #numpy log is really ln
#Plot a histogram of the values:
n,bins,patches=pl.hist(y,bins=100,color='gray')
pl.xlabel('exponential random number')
pl.ylabel('frequency')
pl.title('10,000 exponentially distributed numbers')
#plot the corresponding exponential (normalized by the number of y values)
xi=np.linspace(0,y.max(),100)
norm = y.size/(bins[1]-bins[0])/tau
pl.plot(xi,norm*np.exp(-xi/tau),color='red',lw=2,label="exponential")
pl.legend()
pl.show()
\end{lstlisting}
\capfig{0.5\textwidth}{figures/LCGexpo.png}{\label{fig:LCGexpo} Distribution of 10,000 exponentially distributed random numbers obtained from the numbers in Figure \ref{fig:LCG}.}

In practice, we generally do not need to write our own pseudo-random number generators, and we can usually use built-in functions to generate random numbers according to common distributions. For example, the \code{numpy.random} module in python has many distributions already available, as illustrated in the following code, with histograms of the various random numbers shown in Figure \ref{fig:randomdist}.
\begin{lstlisting}[frame=single] 
nr = 10000 # number of random numbers to generate
#Uniform distribution:
xmin=0
xmax=10
xunif = np.random.uniform(xmin,xmax,nr)
#Normal distribution:
mu = 5
sigma = 2
xnorm = np.random.normal(mu,sigma,nr)
#Binomial distribution:
N=10
p=0.5
xbinom = np.random.binomial(N,p,nr)
#Poisson distribution:
n=3
xpoiss = np.random.poisson(n,nr)
#Plot them all:
bins = np.linspace(0,10,50)
pl.hist(xunif,bins=bins,color='red',alpha=0.5,label='uniform')
pl.hist(xnorm,bins=bins,color='blue',alpha=0.5,label='normal')
pl.hist(xbinom,bins=bins,color='black',alpha=0.5,label='binomial')
pl.hist(xpoiss,bins=bins,color='orange',alpha=0.5,label='poisson')
pl.legend()
pl.show()
\end{lstlisting}
\capfig{0.5\textwidth}{figures/randomdist.png}{\label{fig:randomdist} Histogram of 10,000 random numbers generated in python with uniform, normal, binomial, and poisson distributions.}

\section{Monte Carlo simulations}
Let us perform a Monte Carlo simulation of tossing a coin ten times. One ``experiment'' is to throw the coin ten times, and to count the number of heads that we obtain. Over many experiments, we know that the number of heads that we obtain will be binomially distributed. Let us suppose that our coin in unfair and has a probability $p=0.7$ of landing on heads. If we do not know about the binomial distribution, we can use Monte Carlo simulation to obtain the distribution of the number of times that we get heads in ten coin tosses.

To simulate a single coin toss, we can generate a random number, $r$, that is uniformly distributed between 0 and 1. We can define the coin toss to be heads if $r<p=0.7$ and tails if $r\geq p= 0.7$. Since $r$ is uniform, on average we will get 70\% heads and 30\% tails, as desired. To simulate counting the number of heads in ten coin tosses (a single experiment), we generate ten values of $r$ and count how many of them are smaller than 0.7, as in the python code below:
\begin{lstlisting}[frame=single] 
#Simulation of a set of 10 coin tosses:
#Make 10 random numbers between 0 and 1:
N=10
r = np.random.uniform(0.0,1.0,N)
#Count how many are smaller than 0.7
nheads = r[r<0.7].size
print(nheads," heads in 10 tosses")
\end{lstlisting}
The output is (although it will be random):
\begin{verbatim}
6  heads in 10 tosses
\end{verbatim}

We can now simulate this experiment many times (say 10,000), and obtain the distribution of the number of heads that we get, as in the code below:
\begin{lstlisting}[frame=single] 
import scipy.stats as stats
Nexp=10000 # repeat the experiment 10,000 times
nheads = np.zeros(Nexp) # array to hold the result of each experiment
N=10
#Conduct the Nexp experiments:
for i in range(Nexp):
    #A single experiment
    r = np.random.uniform(0.0,1.0,N)
    #Count how many are smaller than 0.7, add to array of results
    nheads[i] = r[r<0.7].size    
    
#Plot the result:
bins=np.linspace(-0.5,N+0.5,N+2)
pl.hist(nheads,bins=bins, color='gray')
pl.xlabel("number of heads in 10 tosses")
pl.ylabel("frequency")
#Plot the binomial distribution
norm = Nexp/(bins[1]-bins[0]) #normalization, since the histogram is not normalized
xi=np.arange(N+1)
pl.plot(xi,norm*stats.binom.pmf(xi,N,0.7),color='black',lw=3,label='binomial pmf')
pl.legend(loc='best')
pl.show()
\end{lstlisting}
Figure \ref{fig:coinMC} shows a histogram of the number of heads obtained in the 10,000 experiments. The corresponding binomial distribution, with $p=0.7$ and $N=10$ is shown and seen to describe the results of the Monte Carlo simulation, as expected. It is worth noting that we have also effectively implemented an algorithm that can use uniform random numbers to generate binomially-distributed random numbers!

\capfig{0.5\textwidth}{figures/coinMC.png}{\label{fig:coinMC} Histogram of the outcome (the number of heads) of 10,000 coin tosses of an unfair coin with $p=0.7$ of landing heads. Overlaid, is the corresponding binomial distribution, which we expect the results to follow.}

In the previous example, we could have easily calculated the result, since we know from first principles that the distribution of results is binomially distributed. Let us consider a more complicated example, where the distribution of outcomes is not obvious.

Suppose that we are operating a poutine restaurant, and need to decide if it makes sense to stay open one hour later each day. By using Monte Carlo simulation, we can simulate the expected profits of staying open one extra hour to see if they would on average offset the cost of paying to staff the restaurant. We sell 3 types of poutine (traditional, pork, and vegetarian), and each type has a different popularity and results in different amounts of profit (excluding cost of personnel), as tabulated in Table \ref{tab:poutineProfits}.
\begin{table}[h!]
\center
\begin{tabular}{|c|c|c|}
\hline
\textbf{Poutine type} & \textbf{Percent ordered} & \textbf{Profit in \$}\\
\hline
Traditional & 70\% & \$2.00\\
Pork & 20\% & \$2.50\\
Vegetarian & 10\% & \$0.50\\
\hline
\end{tabular}
\caption{\label{tab:poutineProfits} Frequency that each type of poutine is ordered and profit on each type.}
\end{table}

Let us suppose that during a trial period, we have measured that, on average, 15 customers buy poutines in the restaurant during this extra hour. On average, customers buy 1 poutine, but sometimes they buy more (e.g. for their friends). We only count people that enter the store as a customer if they bought at least 1 poutine. For our Monte Carlo simulation, we will consider as a single ``experiment'' determining the profits in one night of staying open for the extra hour. We will then repeat the experiment many times to obtain the distribution of profits that we expect when having the store open for an extra hour. Our strategy for a single experiment is as follows:
\begin{enumerate}
\item Draw a random number, \code{ncustomers}, from a Poisson distribution with a mean 15 to be the number of customers that night.
\item For each customer, draw a random number, \code{npoutines}, from a Poisson distribution with a mean of 1 to be the number of poutines ordered by that customer. If the number is 0, draw again. 
\item For each poutine, draw a uniform random number between 0 and 1, \code{choice}, to represent which type of poutine it is. If \code{choice} is between 0 and 0.7, it is a traditional poutine, if it is between 0.7 and 0.9, it is a pork poutine, and if it is between 0.9 and 1, it is a vegetarian poutine.
\item Add up the profits for all of the poutines, for all of the customers.
\end{enumerate}

The following python code performs this Monte Carlo simulation, giving the distribution of expected profits in a histogram shown in Figure \ref{fig:MCpoutine}.

\begin{lstlisting}[frame=single] 
Nexp = 10000 # the number of experiments (nights)
profits = np.zeros(Nexp)# array to hold the profits for each night
#Perform Nexp experiments
for iexp in range(Nexp):
    #For each night:
    #Generate a random number of customers with (Poisson) mean 15
    ncustomers = np.random.poisson(15)
    #For each customer, simulate their order:
    for icustomer in range(ncustomers):
        #Generate a random number of poutines with Poisson mean of 1
        #Repeat until the number is bigger than 0
        npoutines=0
        while(npoutines<1):
            npoutines = np.random.poisson(1)
        #For each poutine, decide what type it is
        for ipoutine in range(npoutines):
            #Draw a uniform random number between 0 and 1 for the choice
            choice = np.random.uniform()
            #based on the type, add in the profits for that night:
            if choice<0.7: #traditional
                profits[iexp] = profits[iexp]+2.00 
            elif choice>=0.7 and choice <0.9:#pork
                profits[iexp] = profits[iexp]+2.50 
            else:#vegetarian
                profits[iexp] = profits[iexp]+0.50
                
print("Mean profits: ${:.2f}".format(profits.mean())) 
#Plot it:
pl.hist(profits,bins=20,color='gray')
pl.xlabel('profits ($)')
pl.title('Profits simulated for 10,000 nights')
pl.show()
\end{lstlisting}
The output is:
\begin{verbatim}
Mean profits: $46.08
\end{verbatim}
\capfig{0.5\textwidth}{figures/MCpoutine.png}{\label{fig:MCpoutine} Histogram of the distribution of profits in the last hour of the poutine store.}

We can conclude from our simulation that on average, we will generate \$46 of profit by staying open an extra hour, which will be worthwhile if that is enough to cover the cost of the staff. The histogram, if normalized, is truly the probability density function for the expected profits of the store. We can use this pdf to estimate other quantities that may be of interest, such as the profit that we expect to make at lease 90\% of the time (by finding the value where the cumulative distribution function is 0.1). 

Monte Carlo simulations are particularly well-suited for particle physics, where one is often interested in tracking particles. Due to the quantum nature of how particles interact, one cannot definitely predict how a specific particle will interact, but given the distribution of possible interactions, these can be simulated using the Monte Carlo method. For example, to simulate the energy of gamma rays that one would measure with a germanium detector, one would simulate individual gamma rays many times. Some of the gamma rays would never make it to the detector, while some will be absorbed in the detector, while others will scatter and only leave part of their energy in the detector. Simulations involving the tracking of particles are heavily used in medical physics, for example to estimate the radiation dose that a patient will receive in a given configuration.

The Monte Carlo method is a vast topic with many applications, hopefully this brief overview has given you an ideal of the potential of this method to tackle complex problems.

